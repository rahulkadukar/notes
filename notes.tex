
% This file contains notes that I made when I was learning programming languages
\documentclass[11pt,a4paper,oneside]{book}
\usepackage[T1]{fontenc}
\usepackage[margin=1in]{geometry}
\usepackage{listings,caption,color,multicol,textcomp,tabularx,longtable}
\usepackage{tikz}
\usetikzlibrary{shapes.geometric, arrows}

\tikzstyle{startstop} = [
  rectangle,
  rounded corners,
  minimum width=3cm,
  minimum height=0.75cm,
  text centered, 
  draw=black,
  fill=red!30 ]
\tikzstyle{io} = [
  trapezium,
  trapezium left angle=70,
  trapezium right angle=110,
  minimum width=3cm, 
  minimum height=0.75cm,
  text centered,
  draw=black,
  fill=blue!30 ]
\tikzstyle{process} = [
  rectangle,
  minimum width=3cm,
  minimum height=0.75cm,
  text centered,
  draw=black, 
  fill=orange!30 ]
\tikzstyle{decision} = [
  diamond,
  minimum width=3cm,
  minimum height=0.75cm,
  text centered,
  draw=black, 
  fill=green!30 ]
\tikzstyle{arrow} = [
  thick,
  ->,
  >=stealth ]

\setcounter{tocdepth}{1}

\definecolor{keyword}{rgb}{0,0.6,0}
\definecolor{gray}{rgb}{0.5,0.5,0.5}
\definecolor{text}{RGB}{52, 152, 219}
\definecolor{comment}{rgb}{0.58,0,0.82}
\definecolor{string}{RGB}{255,127,0}
\definecolor{caption}{RGB}{188,232,241}
\definecolor{captionText}{RGB}{49,112,143}

\lstset{
  frame=single,
  framexleftmargin=-5.9pt,
  framexrightmargin=2.6pt,
  language=C,
  rulecolor=\color{caption},
  aboveskip=3mm,
  belowskip=3mm,
  showstringspaces=false,
  columns=flexible,
  basicstyle={\small\ttfamily},
  keywordstyle=\color{keyword},
  commentstyle=\itshape\color{comment},
  identifierstyle=\color{text},
  stringstyle=\color{string},
  breaklines=true,
  breakatwhitespace=true,
  tabsize=3
}

\DeclareCaptionFont{white}{\color{captionText}}
\DeclareCaptionFormat{listing}{
  \colorbox{caption}
  {\parbox
    {\dimexpr\textwidth-2\fboxsep\relax}
    {#1#2#3}
  }
}

\captionsetup[lstlisting]{format=listing,labelfont=white,textfont=white}


% This section contains general information about the author
\begin{document}
\title{My Reference Book}
\author{
Rahul Kadukar, \\
North Bergen, New Jersey, \\
USA 07047 \\
\\
\texttt{\textcolor{blue}{rahul.kadukar@rutgers.edu}} \\
\texttt{\textcolor{blue}{kadukar.rahul@gmail.com}}
}
\date{\today}
\maketitle
\begingroup
\let\cleardoublepage\clearpage
\tableofcontents
% This can be added to show listings \lstlistoflistings%
\endgroup

% The programming language C starts here
\part{C}
\chapter{An Introduction}
The reference book that I am using is the 6th edition of \textbf{C Primer Plus}
by \textit{\textbf {Stephen Prata}}.
\section{Hello world}
Hello world program in C.
\noindent \begin{lstlisting}[title=Hello world program, language=C]
  
  /* This is the Hello world program */
  // This is a single line comment

  #include <stdio.h>
  
  int main(){
    printf("Hello world\n");
  }
  
\end{lstlisting}



% The programming language Scala starts here
\part{Scala}
\setcounter{chapter}{0}
\chapter{Types in Scala}
\begin{itemize}
\itemsep-3pt
\item  Byte        8-bit signed two's complement integer (-27 to 27 - 1, inclusive)
\item  Short       16-bit signed two's complement integer (-215 to 215 - 1, inclusive)
\item Int         32-bit signed two's complement integer (-231 to 231 - 1, inclusive)        
\item Long        64-bit signed two's complement integer (-263 to 263 - 1, inclusive)
\item Char        16-bit unsigned Unicode character (0 to 216 - 1, inclusive)        
\item String        a sequence of Chars        
\item Float        32-bit IEEE 754 single-precision float
\item Double        64-bit IEEE 754 double-precision float        
\item Boolean        true or false
\end{itemize} 
\end{document}
